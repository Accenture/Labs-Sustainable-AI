\paragraph{YN studies in the group ``analytical estimation model''} 
 \label{tab:YN-analytical-estimation-model} 
\begin{longtable}{|>{\raggedright\arraybackslash}p{2.75cm}|p{11.75cm}|p{0.7cm}|}
\toprule
\bfseries study & \bfseries detail & \bfseries cites \\
\midrule 
\endhead
\cite{liu2017}, 2017, \acrshort{nu} & formula based on the amount of consumed energy represented by the static features of source codes and hardware specifications, for a computing task in the cloud, accounts for CPU, RAM and Disk & 13 \\
\cite{acar2016c}, 2016, TEEC &  (Tool to Estimate Energy Consumption) based on information from the Sigar library, for CPU, RAM and Disk & 2 \\
\cite{acar2016b}, 2016, TEEC & model based on hardware utilization, for CPU, RAM, Disk, Network & 12 \\
\cite{park2016}, 2016, \acrshort{nu} & model with utilization rates as inputs, for CPU, RAM, Disk, Mainboard, CPU cooler, Case cooler, and Optical Disc Drive & 3 \\
\cite{acar2016a}, 2016, TEEC & model with hardware utilization as input, for processes, accounts for CPU, RAM and Disk & 29 \\
\cite{noureddine2015}, 2015, E-Surgeon & based on PowerAPI and Jalen, for java classes and methods, accounting for CPU and Network & 80 \\
\cite{noureddine2014a}, 2014, Jalen & based on hardware utilization and power estimation of PowerAPI, for java code on CPU and Network & 16 \\
\cite{peng2013}, 2013, \acrshort{nu} & model based on specific \acrshort{pmc}, for CPU, RAM, Disk, I/O Controller & 11 \\
\cite{bourdon2013}, 2013, PowerAPI & model with hardware utilization, frequency, voltage and specifications as inputs, for processes on CPU, RAM and Disk; \textbf{available:} package, code \href{https://github.com/powerapi-ng/powerapi}{\ref*{link-bourdon2013}} & 94 \\
\cite{basmadjian2011}, 2011, \acrshort{nu} & based on hardware utilization and specifications, for a server, accounting for CPU, RAM, Disk, Mainboard, Network, Fan and Power Supply & 130 \\
\cite{singh2009}, 2009, \acrshort{nu} & model based on \acrshort{pmc} and temperature, for CPU cores & 9 \\
\cite{spellmann2009}, 2009, \acrshort{nu} & based on server power metrics, utilization rates and PUE, for servers & 15 \\
\bottomrule
\end{longtable}
